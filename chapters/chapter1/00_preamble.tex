\acrfull{roc} is research project undertaken by a collaboration between the OPEN Lab Team under the leadership of PI Volker Sorger and the HPCL Team under the leadership of PI Tarek El Ghazawi both based at the George Washington University through the funding of an NSF RAISE Grant working to understanding the physics of, develop software for, and fabricate an analog co-processor implemented in Silicon Photonics, and Optical Metatronics, with the goal of calculating approximate solutions to 4 classes of partial differential equations comprised of Laplace, Poisson, Diffusion, and Wave. The approximate accuracy of these solutions, the time required to attain them, the energy utilized in the computation, and the physical dimension of the fabricated chip are compared against analytically derived, and numerically computed solutions, as well against a previously researched electronic analog co-processor.

\par This research is part of a larger trend punctuated by the termination of the \gls{International Technology Roadmap for Semiconductors} (\acrshort{itrs}) 2.0 report with its final 2015 publication \cite{itrs2.0_2015}. The report published for decades comprised an amalgamation of the opinions of worlds leading semiconductor researchers and industry professionals and is arguably most known for setting expectations for semiconductor \gls{technology node} scaling. However due to the challenges posed by the limits of \gls{Moore's law}, the end of \gls{Dennard scaling}, the slowing of \gls{Koomey's law}, and the limits of \gls{Amdahl's Law}. 

\par The \gls{Institute of Electrical and Electronics Engineers} (\acrshort{IEEE}) has pivoted and released the \gls{International Roadmap For Devices And Systems} (\acrshort{irds}) 2017 edition \cite{irds_2017}. In the report \acrshort{irds} acknowledges that 2D scaling will reach fundamental limits beyond 2020 and as a solution they introduce three distinct eras of scaling, Geometrical (1975-2002), Equivalent (2003 $\sim$ 2024), and 3D Power (2025 $\sim$ 2040) with Equivalent Scaling (2003 $\sim$ 2024) being defined by the "reduction of only horizontal dimension in conjunction with introduction of new materials and new physical effects". With "new vertical structures replacing planar transistors". Our research embraces this paradigm shift and falls within the scope of new materials and new physical effects. As of 2018 companies such as Xilinx have started to design processors with CPU, GPU, and FPGA architectures integrated \cite{vissers2019versal}, with the new question being which architecture is best for which task. The PDE co-processing capabilities of ROC fall within this trend.
