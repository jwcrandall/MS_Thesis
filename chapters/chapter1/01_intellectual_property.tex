\par The concept of an analog electronic mesh based computer applied to approximately solve partial differential equations was initially researched by G. Liebmann and colleagues in the 1950's \cite{G.Liebmann_1950, liebmann1954resistance}. In 2000, a programmable \gls{Very Large Scale Integration} (\acrshort{vlsi}) chip for analog solutions to \acrshort{pde}s was patented \cite{ramirez2000digitally} that can be implemented using discrete components or as an \gls{application-specific integrated circuit} (\acrshort{asic}), hosted by a digital computer. In 2015 members of the \acrfull{hpcl} and \acrfull{open} lab and the filed a patent for an optical implementation of a \acrshort{pde} solving circuit implementation of \acrfull{roc} \cite{patent_20170161417}.

\par Due to the focused time I have spent working on the 4 year (2017-2021) \acrfull{nsf} \acrfull{raise} funded \acrshort{roc} project I have been able to uncover the underlying challenges I have been working to address as well as the ones that I have avoided in my goal of understanding accuracy.The ROC project is a product of larger research trend, partially encompassed by what I believe are two grand challenges. 

\par The first being that CMOS technology surpassed analog alternatives in the 1970s because of its versatility and successful scaling, but decreasing CMOS 2D length scaling is ending during the 2020s, which is creating room for many alternative technologies to be proposed. This research space and funding created by the first challenge has allowed teams to think fundamentally about how different algorithms perform computation. After working on the ROC project, I have come to the conclusion that I am sure many other researchers before me have concerning competitiveness encompassed in the second grand challenge. 

\par Algorithms built into hardware must improve their computational time complexity compared to pure software-based implementations to make up for a loss of versatility, but how much versatility must the hardware provide for the effort to be worthwhile? I realize that algorithms built into hardware can have similar computational time complexity but improved energy consumption compared to pure software, but since I am focusing on accuracy as a metric of performance, I ahve not rigorously explore improvements in energy consumption as a metric of performance.
