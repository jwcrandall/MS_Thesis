\par My thesis is composed of three chapters, Context & Motivation, Mathematics & Physics Underlying the Computation Model, and Future Work. The Context & Motivation chapter discusses the intellectual discoveries the word builds on as well as challenges I am working to understand and address. The deliverable discussed correspond to my contributions to the yearly NSF specified project requirements and the structure of the written thesis. The partial differential equation solution methods cover the different ways one can go about solving a PDE and their limitations. The reconfigurable optical computer operation gives an overview of the physical photonic and metatronic analog coprocessors used to solve the PDEs, and accuracy covers the metric I use to evaluate the performance of the computer.

\par The second chapter, the Mathematics & Physics Underlying the Computational Model contains the meat of my investigation. In a top down approach, I first ask the question where do the computational complexity gains originate from in an analog approach compared to a numerical discrete one? I then analytically derive a solution to a partial differential equation in order to understand where the difficulty in analytical approaches arises from and why an algorithmic analytical approach to pde solutions is challenging. Then I show the standard numerical approach to a pde solution, followed by the original electrical analog approach and then the novel Photonic analog and Metatronic analog approaches. In the future work chapter I touch on what is expected by the NSF for the team to accomplish in the remaining two years as well as further Metatronic work that I am hoping to complete.

\subsection{Year 1 from October 2017 to September 2018}
\par During the first year of the NSF Project I created a branch of the "roc_grid_simulation" GitHub repository, initially developed by Engin Kayraklioglu, named boundaryConditions utilizing Python to generate spice based simulations of the electrical grid used by Lieberman to calculate their accuracy compared to an analytical solution to a Laplace PDE solution that I had derived. In order to better understand Liebmann’s finite difference mapping technique I re derived his work which helped me understand the importance of a node and its neighborhood in terms of this relationships effect on accuracy and how it is altered with the physics of Photonics and Metatronics.

\subsection{Year 2 from October 2018 to September 2019}
\par During the second year I created COMSOL Models of the Metatronic ROC in order to study the effects of network size and density on their role in accuracy, leading to an understanding of the importance of \gls{displacement current density} confinement. In my desire to showcase the possible computational advantages of ROC, I estimated computational time complexities values for the future processor in comparison to an electrical analog and how it could be utilized within a and discrete parallelized implementation.
