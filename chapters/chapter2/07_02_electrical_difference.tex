\subsection{Mapping of Difference Equation Approximation to Electrical Resistance Mesh}\label{section:electricalDifferenceEquation}


\begin{equation}\label{eq:supplementalElectricalLaplace}
  \nabla \cdot \epsilon \nabla \varphi = g 
\end{equation}

The use of an electrical mesh allows us to generate the solution of an approximation of the partial differential equation \ref{eq:supplementalElectricalLaplace} refereed to from here on as the electrical difference equation solution. In the partial difference equation \ref{eq:supplementalElectricalLaplace} where $\epsilon$ is the known scalar function, $\varphi$ is the function, and $g=0$ is  the function relationship  for a time independent Laplace second order partial differential equation. By performing \gls{linear interpolation} on Figure \ref{fig:electrical} and disregarding the higher order terms we say that Equation \ref{eq:supplementalElectricalLaplace} is asymptotically equal to:

\begin{equation}\label{eq:2}
\begin{split}
  \nabla \cdot \epsilon \nabla \varphi \simeq \\
  &\quad 
  \frac{2}{h_1 + h_3} 
  \Big[ 
  	\frac{\epsilon_1}{h_1} ( \varphi(\vec{P_1}) - \varphi(\vec{P_0})) 
  +
  	\frac{\epsilon_3}{h_3} ( \varphi(\vec{P_3}) - \varphi\vec{P_0})) 
  \Big]\\
  &\quad
  + 
    \frac{2}{h_2 + h_4} 
  \Big[ 
  	\frac{\epsilon_2}{h_2} ( \varphi(\vec{P_2}) - \varphi(\vec{P_0})) 
  +
  	\frac{\epsilon_4}{h_4} ( \varphi(\vec{P_4}) - \varphi(\vec{P_0})) 
  \Big] 
\end{split}
\end{equation}

By setting neighboring points equal distant $h = h_1 = h_2 = h_3 = h_4$ and having a constant scalar function $\epsilon = \epsilon_1 = \epsilon_2 = \epsilon_3 = \epsilon_4$

\begin{equation}\label{eq:3}
\begin{split}
  \nabla \cdot \epsilon \nabla \varphi \simeq \\
  &\quad
  \frac{2}{h + h} 
  \Big[ 
  	\frac{\epsilon}{h} ( \varphi(\vec{P_1}) - \varphi(\vec{P_0})) 
  +
  	\frac{\epsilon}{h} ( \varphi(\vec{P_3}) - \varphi(\vec{P_0})) 
  \Big]\\
  &\quad
  +
  \frac{2}{h + h} 
  \Big[ 
  	\frac{\epsilon}{h} ( \varphi(\vec{P_2}) - \varphi(\vec{P_0})) 
  +
  	\frac{\epsilon}{h} ( \varphi(\vec{P_4}) - \varphi(\vec{P_0})) 
  \Big]\\
  &\quad
\end{split}
\end{equation}

By simplifying Eq. \ref{eq:3} you are left with

\begin{equation}\label{eq:4}
\begin{split}
  \nabla^2 \varphi \simeq 
  \frac{1}{h^2} \Big[ \varphi(\vec{P_1}) +  \varphi(\vec{P_2}) +  \varphi(\vec{P_3}) +  \varphi(\vec{P_4}) - 4( \varphi(\vec{P_0}) ) \Big]
\end{split}
\end{equation}


Through the assignment of values and directions in Figure \ref{fig:electrical} and the application of Kirchoff's laws the following applies for $n = 1,n=2,n = 3, n = 4$:

\begin{equation}\label{eq:5}
	I_n = \frac{V_n - V_0}{R_n}
\end{equation}

Kirchoff's laws also show that:

\begin{equation}\label{eq:6}
	\sum_{n=1}^{4} I_n = - I_0 
\end{equation}

Eq. \ref{eq:5} and Eq. \ref{eq:6} result in: 

\begin{equation}\label{eq:7}
	\frac{V_1 - V_0}{R_1} + \frac{V_2 - V_0}{R_1} + \frac{V_3 - V_0}{R_3} + \frac{V_4 - V_0}{R_4} = -I_0
\end{equation}

By defining $R_0$ in the following manner
\begin{equation}\label{eq:8}
	R_1 = R_2 = R_3 = R_4 = h^2R_0
\end{equation}

and by defining the current $I_0$ fed into $P_0$ to be
\begin{equation}\label{eq:9}
	I_0 = -g/R_0
\end{equation}

where $g$ is the function relationship defined in Equation \ref{eq:supplementalElectricalLaplace}, a formal analogy can be shown between the voltages $V_n$ appearing at the junctions and the sough function $\varphi$ by comparing Equation \ref{eq:4} and a redefined Equation \ref{eq:7} through the definitions from Equation \ref{eq:8} and \ref{eq:9} in order show that  

\begin{equation}\label{eq:10}
  \nabla^2 \varphi \simeq  \frac{1}{h^2} \Big[ V_1 + V_2 + V_3 + V_4 - 4V_0 \Big]
\end{equation}

The solution to the function $\varphi$ in the differential equation Equation \ref{eq:supplementalElectricalLaplace} has been approximated through the use of a difference equation. The solution is attained through measurement of voltage values at grid points. All that remains is to set the required boundary conditions to obtain the full solution if $g \equiv 0$ everywhere. If $g \neq 0$ currents from Equation \ref{eq:9} have to be fed into mesh points in order to use the resistive mesh to solve Poisson class \acrshort{pde}s. The resistance network performs the "relaxation technique" automatically and instantaneously for a Laplace equation.