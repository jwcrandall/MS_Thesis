\par The concept of an analog mesh based PDE solving co-processor is interesting to anyone who wants to improve the computational time complexity of their algorithm regardless of the physical phenomena utilized to perform by maintaining analog operations in constant time $O(1)$ as shown in Figure \ref{fig:2_01_computational_time_complexity}.
    
\par In order to make up for an analog computers lack a flexibility compared to a digital transistor CMOS based architecture, the analog alternative must also provide flexibility of problem types within its problem domain. This is accomplished through the use of resistors, capacitors, and inductors electrically, and the real and imaginary components of relative permittivity at material interfaces metatronically as shown in Figure \ref{fig:01_02_electrical_photoinc_metatronic}.

\par In the investigation of the photonic and metatronic implementations of ROC, we note that the metatronic architecture exhibits increasing accuracy with increasing node density as shown in Figure \ref{fig:1_03b_mt_accuracy.png}, in the same fashion as the electric node density as shown in Figure \ref{fig:1_03a_elec_accuracy}. However the metatronic circuit has an optical refresh rate in the Terahertz as opposed to the electrical Gigahertz refresh rate. This means that the upper limit to the write time, or clock speed, for the Metatronic circuit is higher than that of the electric circuit because the metatronic circuit will remain in the lumped element domain for higher frequencies than the electric circuit while still abiding by Shannon-Hartley theorem limit. 
    
\par The large size of the photonic mesh and its relatively short optical operation wavelength results in a distributed circuit, changing the physics of optical node splitting from neighborhood defined to geometrically defined, as shown the Figure \ref{fig:01_02b_physics_table}, and thus causing decreased accuracy with increased photonic node density as noted in Figure \ref{fig:1_03b_mt_accuracy.png}.
    
\par The Metatronic solution operates at the same short operation wavelenght as the Photonic circuit, but due to the metatronic circuits nanoscale dimension it operates as a lumped element and thus retains its increasing density increasing accuracy behavior, while still allowing for pde reconfiguration utilizing metatronic capacitance and inductance due to the change in sign of real component of the relative permittivity at material interfaces. Metatronics retains, the computational complexity shown in Figure \ref{fig:1_03b_mt_accuracy.png}, accuracy density behavior shown in Figure \ref{fig:1_03b_mt_accuracy.png}, and reconfigurability advantages of electronics while allowing for an increased upper limit write time, or clock speed. This novelty merits the fabrication of metatronic ROC.

