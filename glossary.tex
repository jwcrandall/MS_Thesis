\newacronym{roc}{ROC}{Reconfigurable Optical Computer}
\newglossaryentry{Reconfigurable Optical Computer}
{
        name= Reconfigurable Optical Computer,
        description={A research project undertaken by a collaboration between the \arcfull{open} Lab Team under the leadership of PI Volker Sorger and the \arcfull{hpcl} Team under the leadership of PI Tarek El Ghazawi both based at the George Washington University through the funding of an NSF RAISE Grant working to understanding the physics of, develop software for, and fabricate an analog co-processor implemented in a two physical technologies including Silicon Photonics, and Metatronics, with the goal of calculating approximate solutions to 4 classes of partial differential equations comprised of Laplace, Poisson, Diffusion, and Wave. The approximate accuracy of these solutions, the time required to attain them, the energy utilized in the computation, and the physical dimension of the fabricated chip are compared against analytically derived, and numerically computed solutions, as well against a previously researched electronic analog co-processor}
}

\newacronym{roe}{ROE}{Reconfigurable Optical Element}
\newglossaryentry{Reconfigurable Optical Element}
{
        name= Reconfigurable Optical Element,
        description={A general term used to describe the base node of either the Photonic or Metatronic Implementation of ROC which are connected via waveguides in the case of Optical ROC and Air Groves in the Case of Metatronic ROC in order to form the mesh utilized in ROC}
}

\newacronym{open}{OPEN}{Orthogonal Physics Enabled Nanophotonics}

\newacronym{hpcl}{HPCL}{High-Performance Computing Lab}

\newacronym{pde}{PDE}{partial differential equation}
\newglossaryentry{partial differential equation}
{
        name= partial differential equation,
        description={In mathematics, a partial differential equation (PDE) is a differential equation that contains beforehand unknown multivariable functions and their partial derivatives. PDEs are used to formulate problems involving functions of several variables, and are either solved by hand, or used to create a computer model. A special case is ordinary differential equations (ODEs), which deal with functions of a single variable and their derivatives. PDEs can be used to describe a wide variety of phenomena such as sound, heat, diffusion, electrostatics, electrodynamics, fluid dynamics, elasticity, or quantum mechanics. These seemingly distinct physical phenomena can be formalised similarly in terms of PDEs. Just as ordinary differential equations often model one-dimensional dynamical systems, partial differential equations often model multidimensional systems. PDEs find their generalisation in stochastic partial differential equations.}
}

\newacronym{itrs}{ITRS}{International Technology Roadmap for Semiconductors}
\newglossaryentry{International Technology Roadmap for Semiconductors}
{
        name= International Technology Roadmap for Semiconductors,
        description={The International Technology Roadmap for Semiconductors (ITRS) is a set of documents produced by a group of semiconductor industry experts. These experts are representative of the sponsoring organisations which include the Semiconductor Industry Associations of the United States, Europe, Japan, South Korea and Taiwan. The documents produced carry this disclaimer: "The ITRS is devised and intended for technology assessment only and is without regard to any commercial considerations pertaining to individual products or equipment". The documents represent best opinion on the directions of research and time-lines up to about 15 years into the future for the following areas of technology: System drivers/design, Test & test equipment, Front-end processes, Process integration, devices and structures, Radio frequency, analog/mixed-signal technologies, Microelectromechanical systems (MEMS), Photolithography, IC interconnects, Factory integration, Assembly & packaging, Environment, safety & health, Yield enhancement, Metrology, Modeling & simulation, Emerging research devices, and Emerging research materials. As of 2017, ITRS is no longer being updated.}
}

\newglossaryentry{Moore's law}
{
        name= Moore's law,
        description={Moore's law is the observation that the number of transistors in a dense integrated circuit doubles about every two years. The observation is named after Gordon Moore, the co-founder of Fairchild Semiconductor and CEO of Intel, whose 1965 paper described a doubling every year in the number of components per integrated circuit and projected this rate of growth would continue for at least another decade. In 1975, looking forward to the next decade, he revised the forecast to doubling every two years. The period is often quoted as 18 months because of a prediction by Intel executive David House (being a combination of the effect of more transistors and the transistors being faster)}
}

\newglossaryentry{Dennard scaling}
{
        name= Dennard scaling,
        description={Dennard scaling, also known as MOSFET scaling, is a scaling law based on a 1974 paper co-authored by Robert H. Dennard, after whom it is named. Originally formulated for MOSFETs, it states, roughly, that as transistors get smaller, their power density stays constant, so that the power use stays in proportion with area; both voltage and current scale (downward) with length.}
}

\newacronym{IEEE}{IEEE}{Institute of Electrical and Electronics Engineers}
\newglossaryentry{Institute of Electrical and Electronics Engineers}
{
        name= Institute of Electrical and Electronics Engineers,
        description={The Institute of Electrical and Electronics Engineers (IEEE) is a professional association with its corporate office in New York City[3] and its operations center in Piscataway, New Jersey. It was formed in 1963 from the amalgamation of the American Institute of Electrical Engineers and the Institute of Radio Engineers. Today, the organization's scope of interest has expanded into so many related fields, that it is simply referred to by the letters I-E-E-E (pronounced Eye-triple-E), except on legal business documents. As of 2018, it is the world's largest association of technical professionals with more than 423,000 members in over 160 countries around the world. Its objectives are the educational and technical advancement of electrical and electronic engineering, telecommunications, computer engineering, and allied disciplines.}
}

\newglossaryentry{co-processor}
{
        name= co-processor,
        description={A coprocessor is a computer processor used to supplement the functions of the primary processor (the CPU). Operations performed by the coprocessor may be floating point arithmetic, graphics, signal processing, string processing, cryptography or I/O interfacing with peripheral devices. By offloading processor-intensive tasks from the main processor, coprocessors can accelerate system performance.}
}

\newglossaryentry{Photonic}
{
        name= Photonic,
        description={Photonics is the physical science of light (photon) generation, detection, and manipulation through emission, transmission, modulation, signal processing, switching, amplification, and sensing. Though covering all light's technical applications over the whole spectrum, most photonic applications are in the range of visible and near-infrared light. The term photonics developed as an outgrowth of the first practical semiconductor light emitters invented in the early 1960s and optical fibers developed in the 1970s.}
}

\newglossaryentry{Plasmonic}
{
        name= Plasmonic,
        description={Plasmonics is a rapidly developing field at the boundary of physical optics and condensed matter physics. It studies phenomena induced by and associated with surface plasmons—elementary polar excitations bound to surfaces and interfaces of good nanostructured metals \cite{stockman2018roadmap}}
}

\newglossaryentry{Metatronic}
{
        name= Metatronic,
        description={Metamaterial-inspired optical nanocircuitry follows the success of modularization in electronics, individual nanoparticles are treated as lumped circuit elements (for example, nanocapacitors, nanoinductors, and nanoresistors) whose impedance is defined in terms of how the nanoparticle modifies the flux of the displacement current, as a function of the applied electric potential. In addition, in analogy with classical circuit wires, lumped elements in metatronic circuits are usually interconnected via D-dot wires, that is, optical wires designed to confine and “guide” the flow of the displacement current. This methodology enables the design of complex nanoparticle systems by using techniques and tools developed for the design of electronic circuits.\cite{li2016waveguide}}
}

\newglossaryentry{displacement current density}
{
    name= displacement current density,
    description={In electromagnetism, displacement current density is the quantity $\frac{\partial D}{\partial t}$ appearing in Maxwell's equations that is defined in terms of the rate of change of $D$, the electric displacement field. Displacement current density has the same units as electric current density, and it is a source of the magnetic field just as actual current is. However it is not an electric current of moving charges, but a time-varying electric field. In physical materials (as opposed to vacuum), there is also a contribution from the slight motion of charges bound in atoms, called dielectric polarization.}
}


\newglossaryentry{technology node}
{
        name= technology node,
        description={The technology node (also process node, process technology or simply node) refers to a specific semiconductor manufacturing process and its design rules. Different nodes often imply different circuit generations and architectures. Generally, the smaller the technology node means the smaller the feature size, producing smaller transistors which are both faster and more power-efficient. Historically, the process node name referred to a number of different features of a transistor including the gate length as well as M1 half-pitch. Most recently, due to various marketing and discrepancies among foundries, the number itself has lost the exact meaning it once held. Recent technology nodes such as 22 nm, 16 nm, 14 nm, and 10 nm refer purely to a specific generation of chips made in a particular technology. It does not correspond to any gate length or half pitch. Nevertheless, the name convention has stuck and it's what the leading foundries call their nodes.}
}

\newacronym{irds}{IRDS}{International Roadmap For Devices And Systems}
\newglossaryentry{International Roadmap For Devices And Systems}
{
        name= International Roadmap For Devices And Systems,
        description={ This initiative focuses on an International Roadmap for Devices and Systems (IRDS) through the work of roadmap teams closely aligned with the advancement of the devices and systems industries. Led by an international roadmap committee (IRC), International Focus Teams (IFTs) will collaborate in the development of a roadmap, and engage with other segments of the IEEE, such as Rebooting Computing, and related industry communities, in complementary activities to help ensure alignment and consensus across a range of stakeholders, such as, Academia, Consortia, Industry, and National laboratories. IEEE, the world's largest technical professional organization dedicated to advancing technology for humanity, through the IEEE Standards Association (IEEE-SA) Industry Connections (IC) program, supports the IRDS to ensure alignment and consensus across a range of stakeholders to identify trends and develop the roadmap for all of the related technologies in the computer industry.}
}

\newacronym{asic}{ASIC}{application-specific integrated circuit}
\newglossaryentry{application-specific integrated circuit}
{
        name= application-specific integrated circuit,
        description={An application-specific integrated circuit is an integrated circuit (IC) customized for a particular use, rather than intended for general-purpose use.}
}

\newacronym{vlsi}{VLSI}{Very Large Scale Integration}
\newglossaryentry{Very Large Scale Integration}
{
        name= Very Large Scale Integration,
        description={Very-large-scale integration (VLSI) is the process of creating an integrated circuit (IC) by combining hundreds of thousands of transistors or devices into a single chip. VLSI began in the 1970s when complex semiconductor and communication technologies were being developed. The microprocessor is a VLSI device. Before the introduction of VLSI technology most ICs had a limited set of functions they could perform. An electronic circuit might consist of a CPU, ROM, RAM and other glue logic. VLSI lets IC designers add all of these into one chip.}
}

\newacronym{nsf}{NSF}{National Science Foundation}

\newacronym{raise}{RAISE}{Research Advanced by Interdisciplinary Science and Engineering}

\newglossaryentry{Analytic expression}
{
        name= Analytic expression,
        description={An analytic expression (or expression in analytic form) is a mathematical expression constructed using well-known operations that lend themselves readily to calculation. Similar to closed-form expressions, the set of well-known functions allowed can vary according to context but always includes the basic arithmetic operations (addition, subtraction, multiplication, and division), exponentiation to a real exponent (which includes extraction of the nth root), logarithms, and trigonometric functions. However, the class of expressions considered to be analytic expressions tends to be wider than that for closed-form expressions. In particular, special functions such as the Bessel functions and the gamma function are usually allowed, and often so are infinite series and continued fractions. On the other hand, limits in general, and integrals in particular, are typically excluded. If an analytic expression involves only the algebraic operations (addition, subtraction, multiplication, division, and exponentiation to a rational exponent) and rational constants then it is more specifically referred to as an algebraic expression.}
}

\newglossaryentry{Numerical analysis}
{
        name= Numerical analysis,
        description={Numerical analysis is the study of algorithms that use numerical approximation (as opposed to general symbolic manipulations) for the problems of mathematical analysis (as distinguished from discrete mathematics). Numerical analysis naturally finds application in all fields of engineering and the physical sciences, but in the 21st century also the life sciences, social sciences, medicine, business and even the arts have adopted elements of scientific computations. As an aspect of mathematics and computer science that generates, analyzes, and implements algorithms, the growth in power and the revolution in computing has raised the use of realistic mathematical models in science and engineering, and complex numerical analysis is required to provide solutions to these more involved models of the world. Numerical analysis continues the long tradition of practical mathematical calculations, where modern numerical analysis does not seek exact answers, because exact answers are often impossible to obtain in practice. Instead, much of numerical analysis is concerned with obtaining approximate solutions while maintaining reasonable bounds on errors.}
}

\newglossaryentry{finite difference method}
{
        name= finite difference method,
        description={In mathematics, finite-difference methods (FDM) are numerical methods for solving differential equations by approximating them with difference equations, in which finite differences approximate the derivatives. FDMs are thus discretization methods. FDMs convert a linear (non-linear) ODE/PDE into a system of linear (non-linear) equations, which can then be solved by matrix algebra techniques. The reduction of the differential equation to a system of algebraic equations makes the problem of finding the solution to a given ODE ideally suited to modern computers, hence the widespread use of FDMs in modern numerical analysis[1]. Today, FDMs are the dominant approach to numerical solutions of partial differential equations.}
}

\newglossaryentry{Conjugate gradient method}
{
        name= Conjugate gradient method,
        description={In mathematics, the conjugate gradient method is an algorithm for the numerical solution of particular systems of linear equations, namely those whose matrix is symmetric and positive-definite. The \\ conjugate gradient method is often implemented as an iterative algorithm, applicable to sparse systems that are too large to be handled by a direct implementation or other direct methods such as the Cholesky decomposition. Large sparse systems often arise when numerically solving partial differential equations or optimization problems. The conjugate gradient method can also be used to solve unconstrained optimization problems such as energy minimization. It was mainly developed by Magnus Hestenes and Eduard Stiefel who programmed it on the Z4. The biconjugate gradient method provides a generalization to non-symmetric matrices. Various nonlinear conjugate gradient methods seek minima of nonlinear equations.}
}

\newglossaryentry{Multigrid method}
{
        name= Multigrid method,
        description={In mathematics, the conjugate gradient method is an algorithm for the numerical solution of particular systems of linear equations, namely those whose matrix is symmetric and positive-definite. The conjugate gradient method is often implemented as an iterative algorithm, applicable to sparse systems that are too large to be handled by a direct implementation or other direct methods such as the Cholesky decomposition. Large sparse systems often arise when numerically solving partial differential equations or optimization problems. The conjugate gradient method can also be used to solve unconstrained optimization problems such as energy minimization. It was mainly developed by Magnus Hestenes and Eduard Stiefel who programmed it on the Z4. The biconjugate gradient method provides a generalization to non-symmetric matrices. Various nonlinear conjugate gradient methods seek minima of nonlinear equations.}
}

\newglossaryentry{Big O notation}
{
        name= Big O notation,
        description={Big O notation is a mathematical notation that describes the limiting behavior of a function when the argument tends towards a particular value or infinity. It is a member of a family of notations invented by Paul Bachmann, Edmund Landau, and others, collectively called Bachmann–Landau notation or asymptotic notation. In computer science, big O notation is used to classify algorithms according to how their running time or space requirements grow as the input size grows. In analytic number theory, big O notation is often used to express a bound on the difference between an arithmetical function and a better understood approximation; a famous example of such a difference is the remainder term in the prime number theorem. Big O notation characterizes functions according to their growth rates: different functions with the same growth rate may be represented using the same O notation. The letter O is used because the growth rate of a function is also referred to as the order of the function. A description of a function in terms of big O notation usually only provides an upper bound on the growth rate of the function. Associated with big O notation are several related notations, using the symbols $o$, $\Omega$, $\omega$, and $\Theata$, to describe other kinds of bounds on asymptotic growth rates.}
}

\newacronym{spice}{SPICE}{Simulation Program with Integrated Circuit Emphasis}
\newglossaryentry{SPICE}
{
        name= SPICE,
        description={SPICE ("Simulation Program with Integrated Circuit Emphasis") is a general-purpose, open-source analog electronic circuit simulator. It is a program used in integrated circuit and board-level design to check the integrity of circuit designs and to predict circuit behavior.}
}

\newglossaryentry{Lumerical INTERCONNECT}
{
        name= Lumerical INTERCONNECT,
        description={INTERCONNECT, Lumerical’s photonic integrated circuit simulator, verifies multimode, bidirectional, and multi-channel PICs. Creating your project in our hierarchical schematic editor, you can use our extensive library of primitive elements, as well as foundry-specific PDK elements, to perform analysis in the time or frequency domain.}
}

\newglossaryentry{COMPSOL Multiphysics}
{
        name=COMPSOL Multiphysics,
        description={COMSOL Multiphysics is a cross-platform finite element analysis, solver and multiphysics simulation software. It allows conventional physics-based user interfaces and coupled systems of partial differential equations (PDEs). COMSOL provides an IDE and unified workflow for electrical, mechanical, fluid, and chemical applications. An API for Java and LiveLink for MATLAB may be used to control the software externally, and the same API is also used via the Method Editor.}
}

\newglossaryentry{post-Moore}
{
        name=post-Moore,
        description={The (CMOS) world is ending next decade, so says the international technology roadmap for semiconductors. In the long term (~2017 THROUGH 2024) ,while power consumption is an urgent challenge, its leakage or static component will become a major industry crisis in the long term, threatening the survival of CMOS technology itself, just as bipolar technology was threatened and eventually disposed of decades ago. [ITRS 2009/2010]. Unlike the situation at the end of the bipolar era, no technology is waiting in the wings. The technological barriers that need to be overcome include new materials and new structures. Materials such as III-V or germanium thin channels on silicon, or even semiconductor nanowires, carbon nanotubes, graphene or others may be needed. Three-dimensional architecture, such as vertically stackable cell arrays in monolithic integration, with acceptable yield and performance. \cite{bluewaters}}
}

\newacronym{mimo}{MIMO}{multiple-input and multiple-output}
\newglossaryentry{multiple-input and multiple-output}
{
        name=multiple-input and multiple-output,
        description={In radio, multiple-input and multiple-output, \\ or MIMO, is a method for multiplying the capacity of a radio link using multiple transmission and receiving antennas to exploit multipath propagation. MIMO has become an essential element of wireless communication standards including IEEE 802.11n (Wi-Fi), IEEE 802.11ac (Wi-Fi), HSPA+ (3G), WiMAX (4G), and Long Term Evolution (LTE 4G). More recently, MIMO has been applied to power-line communication for 3-wire installations as part of ITU G.hn standard and HomePlug AV2 specification.}
}

\newacronym{ito}{ITO}{Indium tin oxide}
\newglossaryentry{Indium tin oxide}
{
        name=Indium tin oxide,
        description={Indium tin oxide (ITO) is a ternary composition of indium, tin and oxygen in varying proportions. Depending on the oxygen content, it can either be described as a ceramic or alloy. Indium tin oxide is typically encountered as an oxygen-saturated composition with a formulation of 74\% In, 18\% O2, and 8\% Sn by weight. Oxygen-saturated compositions are so typical, that unsaturated compositions are termed oxygen-deficient ITO. It is transparent and colorless in thin layers, while in bulk form it is yellowish to grey. In the infrared region of the spectrum it acts as a metal-like mirror. Indium tin oxide is one of the most widely used transparent conducting oxides because of its two main properties: its electrical conductivity and optical transparency, as well as the ease with which it can be deposited as a thin film. As with all transparent conducting films, a compromise must be made between conductivity and transparency, since increasing the thickness and increasing the concentration of charge carriers increases the material's conductivity, but decreases its transparency. Thin films of indium tin oxide are most commonly deposited on surfaces by physical vapor deposition. Often used is electron beam evaporation, or a range of sputter deposition techniques.}
}

\newglossaryentry{probe card}
{
        name=probe card,
        description={A probe card is an interface between an electronic test system and a semiconductor wafer. Typically the probe card is mechanically docked to a prober and electrically connected to a tester. Its purpose is to provide an electrical path between the test system and the circuits on the wafer, thereby permitting the testing and validation of the circuits at the wafer level, usually before they are diced and packaged. It consists, normally, of a printed circuit board (PCB) and some form of contact elements, usually metallic, but possibly of other materials as well.}
}

\newacronym{er}{ER}{extinction ratio}
\newglossaryentry{extinction ratio}
{
        name=extinction ratio,
        description={The ratio $r_e = \frac{P_1}{P_0}$ of two optical power levels of a digital signal generated by an optical source expressed as a fraction in $\si{dB}$ or as percentage.}
}

\newacronym{bc}{BC}{boundary conditions}
\newglossaryentry{boundary conditions}
{
        name=boundary conditions,
        description={a condition that is required to be satisfied at all or part of the boundary of a region in which a set of differential equations is to be solved.}
}

\newglossaryentry{virtualization}
{
        name= virtualization,
        description={The act of creating a virtual (rather than actual) version of something, including virtual computer hardware platforms, storage devices, and computer network resources}
}

\newglossaryentry{distributed shared memory}
{
        name= distributed shared memory,
        description={In computer science, distributed shared memory (DSM) is a form of memory architecture where physically separated memories can be addressed as one logically shared address space. Here, the term "shared" does not mean that there is a single centralized memory, but that the address space is "shared" (same physical address on two processors refers to the same location in memory)}
}

\newglossaryentry{shift register}
{
        name= shift register,
        description={In digital circuits, a shift register is a cascade of flip flops, sharing the same clock, in which the output of each flip-flop is connected to the 'data' input of the next flip-flop in the chain, resulting in a circuit that shifts by one position the 'bit array' stored in it, 'shifting in' the data present at its input and 'shifting out' the last bit in the array, at each transition of the clock input.}
}

\newacronym{dac}{DAC}{digital-to-analog converter}
\newglossaryentry{digital-to-analog converter}
{
        name= digital-to-analog converter ,
        description={In electronics, a digital-to-analog converter (DAC, D/A, D2A, or D-to-A) is a system that converts a digital signal into an analog signal. An analog-to-digital converter (ADC) performs the reverse function.}
}

\newacronym{adc}{ADC}{analog-to-digital converter}
\newglossaryentry{analog-to-digital converter}
{
        name= digital-to-analog converter ,
        description={In electronics, an analog-to-digital converter (ADC, A/D, or A-to-D) is a system that converts an analog signal into a digital signal.}
}

\newglossaryentry{linear interpolation}
{
        name= linear interpolation,
        description={In mathematics, linear interpolation is a method of curve fitting using linear polynomials to construct new data points within the range of a discrete set of known data points}
}

\newglossaryentry{Amdahl's Law}
{
        name= Amdahl's Law,
        description={In computer architecture, Amdahl's law (or Amdahl's argument) \\ $S_{latency}(s)=\frac{1}{(1-p)+\frac{p}{s}}$, where $S_{latency}$ is the theoretical speedup of the execution of the whole task, $s$ is the speedup of the part of the task that benefits from improved system resources, and $p$ is the proportion of execution time that the part benefiting from improved resources originally occupied. The formula gives the theoretical speedup in latency of the execution of a task at fixed workload that can be expected of a system whose resources are improved. It is named after computer scientist Gene Amdahl, and was presented at the AFIPS Spring Joint Computer Conference in 1967. Amdahl's law is often used in parallel computing to predict the theoretical speedup when using multiple processors. For example, if a program needs 20 hours using a single processor core, and a particular part of the program which takes one hour to execute cannot be parallelized, while the remaining 19 hours (p = 0.95) of execution time can be parallelized, then regardless of how many processors are devoted to a parallelized execution of this program, the minimum execution time cannot be less than that critical one hour. Hence, the theoretical speedup is limited to at most 20 times ($\frac{1}{1-P}=20$) For this reason, parallel computing with many processors is useful only for highly parallelizable programs.}
}

\newacronym{ode}{ODE}{ordinary differential equation}
\newglossaryentry{ordinary differential equation}
{
        name= ordinary differential equation,
        description={In mathematics, an ordinary differential equation (ODE) is a differential equation containing one or more functions of one independent variable and the derivatives of those functions. The term ordinary is used in contrast with the term partial differential equation which may be with respect to more than one independent variable.}
}

\newglossaryentry{Koomey's law}
{
        name= Koomey's law,
        description={Koomey's law describes a long-term trend in the history of computing hardware. The number of computations per joule of energy dissipated has been doubling approximately every 1.57 years. This trend has been remarkably stable since the 1950s ($R^2$ of over $98\%$ ) and has been somewhat faster than Moore's law. Jonathan Koomey articulated the trend as follows: "at a fixed computing load, the amount of battery you need will fall by a factor of two every year and a half.}
}

\newglossaryentry{Nanophotonic}
{
        name= Nanophotonics,
        description={Nanophotonics or nano-optics is the study of the behavior of light on the nanometer scale, and of the interaction of nanometer-scale objects with light. It is a branch of optics, optical engineering, electrical engineering, and nanotechnology. It often (but not exclusively) involves metallic components, which can transport and focus light via surface plasmon polaritons. The term "nano-optics", just like the term "optics", usually refers to situations involving ultraviolet, visible, and near-infrared light (free-space wavelengths from 300 to 1200 nanometers).}
}
 
\newglossaryentry{diffusion current}
{
        name= Diffusion current,
        description={Diffusion Current is a current in a semiconductor caused by the diffusion of charge carriers (holes and/or electrons). This is the current which is due to the transport of charges occurring because of non-uniform concentration of charged particles in a semiconductor. The drift current, by contrast, is due to the motion of charge carriers due to the force exerted on them by an electric field. Diffusion current can be in the same or opposite direction of a drift current. The diffusion current and drift current together are described by the drift–diffusion equation. It is necessary to consider the part of diffusion current when describing many semiconductor devices. For example, the current near the depletion region of a p–n junction is dominated by the diffusion current. Inside the depletion region, both diffusion current and drift current are present. At equilibrium in a p–n junction, the forward diffusion current in the depletion region is balanced with a reverse drift current, so that the net current is zero. The diffusion constant for a doped material can be determined with the Haynes – Shockley experiment. Alternatively, if the carrier mobility is known, the diffusion coefficient may be determined from the Einstein relation on electrical mobility.}
}

\newglossaryentry{optical intensity}
{
        name= optical intensity,
        description={The optical intensity I, e.g. of a laser beam, is the optical power per unit area, which is transmitted through an imagined surface perpendicular to the propagation direction. The units of the optical intensity (or light intensity) are W/m2 or (more commonly) W/cm2. The intensity is the product of photon energy and photon flux. It is sometimes called optical energy flux.}
}

\newglossaryentry{RC time constant}
{
        name= RC time constant,
        description={The RC time constant, also called tau, the time constant (in seconds) of an RC circuit, is equal to the product of the circuit resistance (in ohms) and the circuit capacitance (in farads), i.e. $\tau = RC$}
}

\newglossaryentry{green computing}
{
        name= RC time constant,
        description={The RC time constant, also called tau, the time constant (in seconds) of an RC circuit, is equal to the product of the circuit resistance (in ohms) and the circuit capacitance (in farads), i.e. $\tau = RC$}
}

\newglossaryentry{distributed element model}
{
        name= distributed element model,
        description={In electrical engineering, the distributed element model or transmission line model of electrical circuits assumes that the attributes of the circuit (resistance, capacitance, and inductance) are distributed continuously throughout the material of the circuit. This is in contrast to the more common lumped element model, which assumes that these values are lumped into electrical components that are joined by perfectly conducting wires. In the distributed element model, each circuit element is infinitesimally small, and the wires connecting elements are not assumed to be perfect conductors; that is, they have impedance. Unlike the lumped element model, it assumes non-uniform current along each branch and non-uniform voltage along each node. The distributed model is used at high frequencies where the wavelength becomes comparable to the physical dimensions of the circuit, making the lumped model inaccurate}
}

\newacronym{nic}{NIC}{Nanofabrication and Imaging Center}

\newacronym{aim photonics}{AIM Photonics}{American Institute for Manufacturing Integrated Photonics}

\newglossaryentry{passive optical}
{
        name= passive optical,
        description={A passive optical networks (PON) distinguishing feature is that it implements a point-to-multipoint architecture, in which unpowered fiber optic splitters are used to enable a single optical fiber to serve multiple end-points.}
}

\newacronym{evl}{EVL}{epsilon very large}
\newacronym{enz}{ENZ}{epsilon-near-zero}
\newglossaryentry{epsilon-near-zero}
{
        name=epsilon-near-zero,
        description={The window of frequency in which the permittivity is low, i.e., near the plasma frequency, has also become a topic of research interest in several potential applications. The first attempts to build a material with low permittivity at micro- wave frequencies date back to several decades ago, where their use was proposed in antenna applications for enhancing the radiation directivity. Similar attempts have been presented over the past years with analogous purposes. Other more recent investigations of the properties of $\epsilon$-near-zero (ENZ) materials and metamaterials and their intriguing wave interaction properties have been reported. In particular the possibility of designing a bulk material impedance matched with free space but whose permittivity and perme- ability are simultaneously very close to zero. The main attempt has been to exploit the low-wave-number (index near zero) propagation in such materials, which might provide a relatively small phase variation over a physically long distance in these media. When interfaced with materials with larger wave number, this implies the presence of a region of space with almost uniform phase distribution, providing the possibility for directive radiation toward the broadside to a planar interface. The possibility of utilizing the matched low-index metamaterial for transforming curved phase fronts into planar ones has been suggested, exploiting the matching be-tween the aforementioned metamaterial and free space. Nader Engheta's group has proposed several different potential applications of ENZ and/or $\mu$-near-zero materials for different purposes. Relying on the directivity enhancement that such materials may provide.\cite{alu2007epsilon}}
}

\newglossaryentry{lumped element model}
{
        name=lumped element,
        description={The lumped element model (also called lumped parameter model, or lumped component model) simplifies the description of the behaviour of spatially distributed physical systems into a topology consisting of discrete entities that approximate the behaviour of the distributed system under certain assumptions. It is useful in electrical systems (including electronics), mechanical multibody systems, heat transfer, acoustics, etc. Mathematically speaking, the simplification reduces the state space of the system to a finite dimension, and the partial differential equations (PDEs) of the continuous (infinite-dimensional) time and space model of the physical system into ordinary differential equations (ODEs) with a finite number of parameters.}
}

\newglossaryentry{Drude model}{
    name=Drude model,
    description={The Drude model of electrical conduction was proposed in 1900 by Paul Drude to explain the transport properties of electrons in materials (especially metals). The model, which is an application of kinetic theory, assumes that the microscopic behavior of electrons in a solid may be treated classically and looks much like a pinball machine, with a sea of constantly jittering electrons bouncing and re-bouncing off heavier, relatively immobile positive ions.}
}

\newglossaryentry{insertion loss}{
    name=insertion loss,
    description={}
}
